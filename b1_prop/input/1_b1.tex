For a spin-1/2 target and after requiring parity 
and time reversal invariance, only four independent helicity amplitudes are 
necessary to describe virtual Compton scattering. This number doubles in the case 
of a spin-1 target as the spin can be in three states (+, 0, -). The 
hadronic tensor can therefore be decomposed into eight independent structure 
functions:
%
\begin{eqnarray}
W_{\mu\nu} &=& - F_1 g_{\mu\nu} + F_2 \frac{P_{\mu} P{\nu}}{\nu} \nonumber \\
          & & - b_1 r_{\mu\nu} + \frac{1}{6} b_2 (s_{\mu\nu} + t_{\mu\nu} + u_{\mu\nu}) \nonumber \\
          & & + \frac{1}{2} b_3 (s_{\mu\nu} - u_{\mu\nu}) + \frac{1}{2} b_4 (s_{\mu\nu} - t_{\mu\nu}) \nonumber \\
          & & + i \frac{g_1}{\nu} \epsilon_{\mu\nu\lambda\sigma} q^{\lambda} s^{\sigma} 
              + i \frac{g_2}{\nu^2} \epsilon_{\mu\nu\lambda\sigma} q^{\lambda} (p \cdot qs^{\sigma}  
              - s \cdot qp^{\sigma})
\label{had-tensor}
\end{eqnarray}
%
The expressions of $r_{\mu\nu}$, $s_{\mu\nu}$, $t_{\mu\nu}$ and $u_{\mu\nu}$ can be 
found in~\cite{Hoodbhoy:1988am}. They all contain terms proportional to the 
polarization of the target $E$. The structure functions $F_1$, $F_2$, $g_1$ and 
$g_2$ have the same expressions and are measured the same way as for a spin-1/2 
target. The spin-dependent structure functions $b_1$, $b_2$, $b_3$, $b_4$ are 
symmetric under $\mu\leftrightarrow\nu$ and $E\leftrightarrow E^*$ and therefore can 
be isolated from $F_1$ and $g_1$ by unpolarized beam scattering of a polarized 
spin-1 target.
%%%%%%%%%%%%%
%%%%%%%%%%%%%
\subsection{The Operator Product Expansion}
%
In the Operator Product Expansion (OPE) framework, the leading operators 
$O_V^{\mu_1...\mu_n}$ and $O_A^{\mu_1...\mu_n}$ in the expansion are twist two. For a 
spin-1 target, the matrix elements of the time-ordered product of two currents 
$T_{\mu\nu}$ have the following expressions:
%
\begin{eqnarray}
<p,E|O_V^{\mu_1...\mu_n}|p,E>&=&S[a_np^{\mu_1}...p^{\mu_n}+d_n(E^{*\mu_1}E^{\mu_2}-\frac{1}{3}p^{\mu_1}
p^{\mu_2})p^{\mu_3}...p^{\mu_n}], \nonumber \\
<p,E|O_A^{\mu_1...\mu_n}|p,E>&=&S[r_n\epsilon^{\lambda\sigma\tau\mu_1}E_{\lambda}^*E_{\sigma}p_{\tau}
p^{\mu_2})...p^{\mu_n}]
\label{matrix-elt}
\end{eqnarray}
%
The non-zero value of $b_1$ arises from the fact that, in a spin-1 target, the 
$\frac{1}{3}p^{\mu_1}p^{\mu_2}$ doesn't cancel the tensor structure $E^{*\mu_1}E^{\mu_2}$. 
The coefficient $d_n$ can be extracted from the comparison of $T_{\mu\nu}$ expansion 
and the spin-1 target hadronic tensor Eq.~\ref{had-tensor} as follows:
%
\begin{eqnarray}
b_1(\omega)&=&\sum_{n=2,4,...}^\infty 2 C_n^{(1)} d_n \omega^n, \nonumber \\
b_2(\omega)&=&\sum_{n=2,4,...}^\infty 4 C_n^{(2)} d_n \omega^{n-1},
\end{eqnarray}
%
for $1 \le |\omega| \le \infty$ (where $\omega = 1/x$). A Callan-Gross-type relation 
exists for between the two leading order tensor structure functions:
%
\begin{eqnarray}
 2 x b_1 = b_2
\label{callan-gross}
\end{eqnarray}
%
valid at lowest order of QCD, where $C_n^{(1)} = C_n^{(2)}$.
At higher orders, Eq.~\ref{callan-gross} is violated.

Sum rules can be 
extracted from the moments of the tensor structure functions:
%
\begin{eqnarray}
\int_0^1 x^{n-1}~b_1(x)~dx &=& \frac{1}{2}~C_n^{(1)}~d_n, \nonumber \\
\int_0^1 x^{n-2}~b_2(x)~dx &=& C_n^{(2)}~d_n,
\label{sr}
\end{eqnarray}
%
where n is even. 

The OPE formalism is based on QCD and is target-independent. However, a target dependence 
is generated by Eq.~\ref{matrix-elt}, and spin-1 structure functions are subject to 
the same QCD corrections and their moments have the same anomalous dimensions as for 
a spin-1/2 target. In addition, the tensor structure functions should exibit the same 
scaling behavior as $F_1$ and $F_2$, since they are generated from the same matrix 
element $O_V^{\mu_1...\mu_n}$.
%%%%%%%%
%%%%%%%%
\subsection{The Parton Model}
%
In the infinite momentum frame\footnote{All spins and
momenta are along the $z$-axis.} of the parton model, 
the scattering of the virtual photon off a free quark 
with spin up (or down), which carrys a momentum fraction $x$ of the spin-$m$ hadron, can be 
expressed through the hadronic tensor $W_{\mu\nu}^{(m)}$:
%
\begin{eqnarray}
W_{\mu\nu}^{(1)} = \Bigg(- \frac{1}{2} g_{\mu\nu} + \frac{x}{\nu} P_{\mu}P{\nu}\Bigg) 
                               \Big(q^1_{\uparrow}(x) + q^1_{\downarrow}(x)\Big) \nonumber
                    + \frac{i \epsilon_{\mu\nu\lambda\sigma} q^{\lambda} s^{\sigma}}{2 \nu} 
                               \Big(q^1_{\uparrow}(x) - q^1_{\downarrow}(x)\Big),
\label{had-tensor-1}
\end{eqnarray}
%
for a target of spin projection equal to 1 along the $z$-direction, and:
%
\begin{eqnarray}
W_{\mu\nu}^{(0)} = \Bigg(- \frac{1}{2} g_{\mu\nu} + \frac{x}{\nu} P_{\mu}P{\nu}\Bigg) 
                               2 q^0_{\uparrow}(x) 
\label{had-tensor-0}
\end{eqnarray}
%
for a target of spin projection equal to zero along the $z$-direction. The tensor 
structure functions $b_1$ and $b_2$ can be expressed from the comparison of 
$W_{\mu\nu}^{(1)} - W_{\mu\nu}^{(0)}$ with Eq.~\ref{had-tensor} as follows:
%
\begin{eqnarray}
b_1(x) &=& \frac{1}{2} \Big( 2 q^0_{\uparrow}(x) - q^1_{\uparrow}(x) - q^1_{\downarrow}(x) \Big) \nonumber \\
b_2(x) &=& 2 x b_1(x)
\label{TSF-parton}
\end{eqnarray}
%
Therefore the tensor structure functions depend of the spin-averaged parton distributions 
$q^1(x) = q^1_{\uparrow}(x) + q^1_{\downarrow}(x)$ and $q^0(x) = q^0_{\uparrow}(x) + q^0_{\downarrow}(x) 
= 2 q^0_{\uparrow}(x)$ (since, by parity, $q^m_{\uparrow}= q^m_{\downarrow}$) and measure the difference
in partonic constituency in an $m$=1 target and an $m$=0 target. 


%The three leading twist structure functions are $F_1$, $g_1$ and $b_1$.
%The tensor structure function $b_1$, which is leading-twist like  $F_1$ and 
%$g_1$, is quite interesting, in that it presents a simple gauge of nuclear 
%effects: $b_1$ would vanish if the deuteron was simply a proton and neutron in 
%a relative S state.
%
%
%%The Hermes collaboration  made a first measurement~\cite{Airapetian:2005cb} of 
%$b_1$ and found significantly non-zero results.  Beyond providing insight into
%nuclear structure, this has the potential to impact $g_1^n$ and $g_2^n$ extractions, 
%where $b_1$ has traditionally been ignored when the neutron is extracted from 
%deuteron data.




