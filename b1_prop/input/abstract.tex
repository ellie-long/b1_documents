Inclusive scattering from a spin-1 target is described by                                     
eight structure functions. Four of these, the so-called tensor structure functions, 
do not exist in the case of a spin-1/2 target.  Until now, tensor structure has been 
largely unexplored, so the study of these quantities holds the potential of 
initiating a new field of spin physics.
In particular, the EMC experiment revealed that only a small fraction
of the nucleon spin is carried by valence quarks.
Two decades later, the `spin crisis' remains an open issue. 
Quark orbital angular momentum is now considered to be one of the principal 
contributions in generating the nucleon spin, but a precise determination of this
critical piece has been elusive. The leading twist tensor structure function
$b_1$ 
%, which characterizes spin-1 hadrons such as the deuteron, 
can provide 
new insight into this puzzle, since $b_1$ can be directly connected to effects arising from 
orbital angular momentum:  $b_1$ vanishes for the case of the deuteron constituents
 in a relative S state.  For this reason, 
it provides a unique tool to study strictly partonic effects, while also being sensitive to 
cumulative nuclear properties, such as the in-medium modification of the nucleon substructure, and the EMC effect.
Depending on the choice of Bjorken variable, two disparate phenomena can be explored via measurement
of $b_1$. At low $x$, shielding/anti-shielding effects are expected to dominate, 
while at  high $x$, $b_1$ can provide one clean way to study `Novel QCD effects';
for example, hidden color due to 6-quark configuration. Since the D-state contribution to the 
deuteron wave function is relatively well known, any novel effects should be 
readily observable.


We propose a measurement of the deuteron tensor asymmetry $A_{zz}$, and extract $b_1$ in the region $0.05<x<0.60$, 
for $1.0<Q^2<5.0$ GeV$^2$.  This  will provide  access to the tensor quark polarization, and allow a test of the 
Close-Kumano sum rule. A polarized solid $^6$LiD target will be utilized along with the Hall A SoLID spectrometer,
and an unpolarized  100 nA, 11 GeV incident beam during a 38 day measurement.



%fundamental quantities $\delta_{\dagger}q$ and $\delta_{\dagger}\overline{q}$





