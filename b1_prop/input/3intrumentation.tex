%\subsection{Beamline Instrumentation}

%\subsection{Beamline Instrumentation}
%\subsubsection{Beam Current and Beam Charge Monitor}
%Beam currents less than 100 nA are typically used with the polarized target in order
%to limit depolarizing effects and large variations in the density.  All three experimental halls 
%are instrumented (or will soon be) with appropriate 

%Standard BCM cavities have a linearity good to 0.2\% for currents ranging from 
%180 down to 1 uA. 
%High accuracy at even lower currents will be possible due to ongoing upgrades,
%which will be complete before this experiment might be scheduled.
%In addition, experiment E05-004\cite{eDprop} has just recently  commissioned a tungsten 
%beam calorimeter, in order to have a good calibration for $I<3 \mu A$.
%Preliminary results show an absolute calibration of the Hall A BCM
%with 1\% accuracy for currents ranging from 3 $\mu A$ down to 0.5$\mu A$.
%The calorimeter will be located just after the first BPM and before the first dipole
%(see Fig.~\ref{beamline}).
%In the worst-case scenario, the tungsten calorimeter will allow at least 
%2\% accuracy~\cite{ARNE} on the charge determination all the way down to 50 nA.
%
%\subsubsection{Beam Polarimetry}
%We will utilize the Moeller polarimeter as part of the standard Hall X equipment. 
%During operation, 
%0.3 to 0.5 $\mu A$ of current are incident on  a foil of
%iron polarized by a magnetic field.
%The expected systematic uncertainty~\cite{Alcorn:2004sb} of the Moeller
%measurement is 3.5\% or better. 
%Moeller runs will be scheduled at least once per energy change.


%The Compton polarimeter normally is used for a continuous non-invasive
%beam polarization monitor.  However, it is not very well suited to run
%at low energy or low current.
%To provide a cross check of the Moller polarimeter, we may dedicate some
%high current beam time (without polarized target) specifically for Compton
%polarimeter measurements.





%\subsection{The Spectrometers}
% 
%\subsubsection{Detector Stack }
%The standard detector stack will be used for detecting electrons. We will require the 
%usual VDC,
%scintillators S1 and S2, the gas Cerenkov and
%pion rejector/shower counter for particle identification.
%Performance of the spectrometers are well known 

%\subsubsection{Optics}
%A study of the change of the optics coming from the target field was done by xxxx.



%\subsubsection{Data Acquisition}
%We will utilize the standard Hall X data acquisition (DAQ) system which is based on Fastbus 
%1877 TDC and Fastbus 1881 ADC. 
%The DAQ will be run in two single arm mode which allows up to 4
%KHz rate of data for each arm.  


%\end{document}
%% add GEn review by J.H Mitchell
%%http://galileo.phys.virginia.edu/~dbd/JHM_beamline.pdf




