The EMC experiment revealed that only a small fraction
of the nucleon spin is carried by quarks.
Two decades later, the spin crisis remains an open issue. 
Quark orbital angular momentum is now considered to be one of the principal 
contributions in generating the nucleon spin, but a precise determination of this
critical piece has remained elusive. The leading twist tensor structure function
$b_1$ of spin-1 hadrons 
can provide 
new insight into this puzzle, since it is directly related to effects arising from 
orbital angular momentum, which differ from the case in a spin-1/2 target.
%:  $b_1$ vanishes for the case of the deuteron constituents in a relative S state.  
For this reason, it provides a unique tool to study partonic effects, while also being sensitive to 
%cumulative nuclear properties, such as the EMC effect and other in-medium modifications to nucleon substructure.
coherent nuclear properties in the simplest nuclear system.

%Inclusive scattering from a spin-1 target is described by     
%eight structure functions. Four of these, the so-called tensor structure functions,
%do not exist in the case of a spin-1/2 target.


%Depending on the choice of Bjorken variable $x$, two separate phenomena can be explored via measurement
%of $b_1$. 
At low $x$, shadowing effects are expected to dominate $b_1$, 
while at larger values, $b_1$ provides a clean probe of exotic QCD effects, such as
hidden color due to 6-quark configuration. Since the 
%D-state contribution to the 
deuteron wave function is relatively well known, any novel effects are expected to be
readily observable.  All available models  predict a small or vanishing value of $b_1$ at moderate $x$.  However, the
first pioneer measurement of $b_1$ at HERMES revealed a crossover to an anomolously large negative value in the region $0.2 <x<0.5$, albeit with relatively large experimental uncertainty.  

We will perform an inclusive measurement of the deuteron polarized cross sections in the
region $\XMIN<x<\XMAX$, for $\QMIN<Q^2<\QMAX$ GeV$^2$.
With \production_days days of 11 GeV incident beam,  we can determine $b_1$ with sufficient 
precision to discriminate between conventional nuclear models, and the more exotic behaviour
which is hinted at by the HERMES data.
The UVa solid polarized \TARGET target will be used, along with the 
Hall C spectrometers, and an unpolarized  \CURRENT nA beam.
An additional \overhead_days days will be needed for overhead.
This measurement  will provide access to the tensor quark polarization, and allow a test of the 
Close-Kumano sum rule, which vanishes in the absence of tensor polarization in the quark 
sea. 
Until now, tensor structure has been largely unexplored, so the study 
of these quantities holds the potential of initiating a new field of spin physics at 
Jefferson Lab.


%fundamental quantities $\delta_{\dagger}q$ and $\delta_{\dagger}\overline{q}$





