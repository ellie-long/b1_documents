\appendix

\section{Feasibility Study in Hall C with the HMS and SHMS Spectrometers}

We studied the feasibility of a measurement on the leading twist tensor structure 
function $b_1$ for $0.15<x<0.55$, $1.5<Q^2<4.2$ GeV$^2$ and $W \ge 2.0$ GeV. Fig.~\ref{kincov} 
shows the kinematic coverage available at JLab utilizing an 11 GeV beam, and the Hall C HMS 
and SHMS spectrometers at forward angle. 

\begin{figure}[h]
\begin{center}
\includegraphics[width=0.6\textwidth]{figs/hallc/cov_split.eps}
\caption{\label{kincov} Kinematic coverage for 11 GeV beam in Hall C using the HMS and SHMS.}
\end{center}
\end{figure}

The polarized target considered in this proposal is the solid lithium deuteride (LiD) 
since it has a better dilution factor than ND$_3$. The vector polarization, packing fraction and 
dilution used in the estimate of the rates are 55\%, 0.55 and 0.50 respectively. With an incident 
electron beam current of \CURRENT nA, the 
expected deuteron luminosity is $2\times 10^{35}$ cm$^{-2}$s$^{-1}$. The momentum bite and the acceptance 
were assumed to be $\Delta P = \pm 8\%$ and $\Delta\Omega = 6.5$ msr for the HMS, and $\Delta P 
= ^{+20\%}_{-8\%}$ and $\Delta\Omega =4.4$ msr for the SHMS. For the choice of the kinematics, 
special attention was taken onto the angular and momentum limits of the spectrometers: for the 
HMS, $10.5^{\circ} \le \theta \le 85^{\circ}$ and $1 \le P_0 \le 7.3$ GeV/c, and for the SHMS, 
$5.5^{\circ} \le \theta \le 40^{\circ}$ and $2 \le P_0 \le 11$ GeV/c. In addition, the 
opening angle between the spectrometers is physically constrained to be larger than 17.5$^{\circ}$. 
The kinematics, rates, projected 
statistical uncertainties of $A_{zz}$ and $b_1$ along with the time necessary to achieve this
precision are summarized in Table~\ref{RATES}. The projected uncertainties are displayed in 
Fig.~\ref{PROJ}. Only rates with kinematics coverage of $W \ge 2.0$ GeV were computed. A 
total of 24 days of beam time is needed for production data.

\begin{figure}
\begin{center}
\includegraphics[width=0.45\textwidth]{figs/hallc/Azz_proj_lin.eps}
\hspace{0.5cm}
\includegraphics[width=0.45\textwidth]{figs/hallc/xb1_proj_newmiller_lin.eps}
\caption{\label{PROJ}
{\bf Left: }
Projected precision of the tensor asymmetry $A_{zz}$ with 30 days of beam time. Also 
shown are the data from Hermes\cite{Airapetian:2005cb} and the calculation from Kumano~\cite{Kumano:2010vz}.
{\bf Right:} Corresponding projected precision of the tensor structure function $b_1$. In addition,
an updated calculation of Ref.~\cite{Miller:1989nc} from Miller is shown~\cite{Miller_tmp}. The black band
represents the systematic uncertainties but a 60\% uncertainty should be added from the contribution
of $F_1$ in the measured cross sections.}
\end{center}
\end{figure}
%
\subsection{Overhead}
%
In order to calibrate the target NMR system, elastic scattering measurement will be performed at an 
incident energy of 2.2 GeV. Two days, including the energy pass change, should be sufficient. The 
measured cross sections will need to be corrected by the dilution from the unpolarized materials
contained in the target and by the packing fraction due to the granular composition of the target 
material. In order to evaluate these two factors, a total of 8 hours of data will be taken on a 
lithium target and on a carbon target. Target annealing and target material change will be done once 
a week. Table~\ref{OVHEAD} summarizes the expected overhead. About 4 additional days will be needed 
for calibration, background study and configuration changes.

\begin{table}[h]
\begin{center}
\caption{\label{OVHEAD}Summary of the overhead}
\begin{tabular}{c|ccc}\hline\hline
     ~                 &  time   & number &   total time \\
     ~                 &  (hrs)  &   ~    &    (hrs)     \\
\hline
elastic                &    ~    &   ~    &     48.0     \\
dilution               &    ~    &   ~    &     8.0      \\
configuration change   &   0.5   &   4    &     2.0      \\
beam energy meas.      &   2.0   &   2    &     4.0      \\
BCM calibration        &   1.0   &   2    &     2.0      \\
target annealing       &   2.5   &   4    &     10.0     \\
target material change &   4.0   &   4    &     16.0     \\
\hline\hline
\end{tabular}
\end{center}
\end{table}

\subsection{Background}
The pion background has not been estimate yet for this measurement, but the detector package
of the HMS and SHMS should provide a sufficient particle identification efficacy.

\subsection{Experimental Method}
Following Ref.~\cite{Hoodbhoy:1988am} it is possible to isolate the tensor structure function 
$b_1$ from the parallel cross section expressed as follows:
\begin{eqnarray}
\frac{d^2\sigma_{\parallel}}{dx dy} = \frac{e^4 M E}{2 \pi Q^4} [1+(1-y)^2] [x F_1(x) + (\frac{2}{3} - H^2) x b_1(x)]
\label{xsmeth}
\end{eqnarray}
%
where $H^2=(P+2)/3$ is the target spin projection along the beam and $P$ is the target 
polarization. The polarized cross sections $d^2\sigma_{\parallel}/d\Omega dE'$ are measured
by scattering an unpolarized electron beam off a spin-1 target polarized longitudinally 
to the electron beam direction. 

From Eq.~\ref{xsmeth}, the beam time needed to achieve an absolute uncertainty 
of $\delta A_{zz}$ can be deduced and is expressed as follows:
\begin{eqnarray}
T = \frac{1}{R_D~f~(P_{zz}~\delta A_{zz})^2}
\label{time-eq}
\end{eqnarray}
where $R_D$ is the deuteron rate, $f$ the dilution factor and $P_{zz}$ the tensor polarization. 
The expression of the tensor asymmetry $A_{zz}$ can be found in Eq.~\ref{Azz}.

\subsection{Conclusion}

The main concern of this method is the systematic uncertainty from $F_1$. 
Since $F_1$ is about 30 times larger than $b_1$ (at $x = 0.45$),  a 2\% uncertainty in $F_1$
will translate into 60\% systematic uncertainty in $b_1$ which is larger than HERMES uncertainty.
We believe any precision measurement will have to come from the asymmetry or ratio method.

\begin{table}
\begin{center}
\caption{\label{RATES}Summary of the kinematics and rates using Hall C HMS (top part of the table) and SHMS (bottom part of the table) spectrometers for the measurement.}
\begin{tabular}{|ccc|cc|c|cc|cc|c|}\hline
 $<x>$  & $<Q^2>$      &  $<W>$  &    $P_0$    &    $\theta$  &  Rates & $A_{zz}$ & $\delta A_{zz}^{stat}$    & $b_1$  & $\delta b_1^{stat}$ & time   \\
  ~     & (GeV)$^2$  & (GeV) & (GeV)  &     (deg.)  &   (Hz)  & $\times 10^{-2}$ & $\times 10^{-2}$ &  $\times 10^{-2}$ & $\times 10^{-2}$    & (hours) \\
\hline\hline
0.15 &   2.0 &    3.5 &   4.22 &   11.9 &   6.34 & -0.51 &  0.23 &  0.71 & 0.32  &   285 \\
0.35 &   2.8 &    2.5 &   6.93 &   11.0 &  14.77 &  0.75 &  0.22 & -0.36 & 0.11  &   128 \\
0.45 &   3.5 &    2.3 &   7.04 &   12.2 &   8.02 &  1.2  &  0.36 & -0.31 & 0.93  &    91 \\
0.55 &   4.2 &    2.1 &   7.11 &   13.3 &   4.03 &  1.6  &  0.54 & -0.19 & 0.065 &    80 \\
\hline\hline
0.15 &   2.0 &    3.5 &   4.22 &   11.9 &   7.03 & -0.51 &  0.23 &  0.71 & 0.32  &   257 \\
0.25 &   2.0 &    2.6 &   6.93 &    9.3 &  41.77 &  0.19 &  0.15 & -0.16 & 0.13  &   101 \\
0.35 &   2.5 &    2.4 &   7.37 &   10.1 &  21.87 &  0.75 &  0.22 & -0.37 & 0.11  &    86 \\
0.55 &   4.2 &    2.1 &   7.11 &   13.3 &   2.43 &  1.6  &  0.54 & -0.19 & 0.065 &   133 \\
\hline\hline
\end{tabular}
\end{center}
\end{table}

%\subsection{Systematic uncertainty}

%In this proposal, we consider using a model for the unpolarized structure function
%$F_1$ needed to extract $b_1$ from Eq.~\ref{xsmeth}. So no time has been attributed to a 
%possible measurement of $F_1$ at each of our kinematics.

%\begin{table}[h]
%\begin{center}
%\caption{\label{sys}Relative systematic uncertainties in the extraction of $b_1$ from the
%longitudinal cross sections.}
%\begin{tabular}{|cc|}\hline
%Polarimetry                  &   8\%   \\
%Dilution/packing fraction    &   5\%   \\
%Acceptance                   &   3\%   \\
%Tracking efficiency          &   1\%   \\
%Trigger  efficiency          &  0.5\%  \\
%PID  efficiency              &  1.5\%  \\
%Charge measurement           &  0.5\%  \\
%Energy measurement           &  0.05\% \\
%Radiative corrections        &   5\%   \\         
%$F_1$ model                  &  10\%   \\
%\hline
%COMBINED UNCERTAINTY         &  15\%   \\
%hline
%\end{tabular}
%\end{center}
%\end{table}
